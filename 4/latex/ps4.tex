\documentclass[12pt, oneside, letterpaper, fleqn]{article}

\usepackage{ducky}
\pagestyle{fancy}
\headheight 1.6cm
\fancyhead[L]{Bryance Oyang\\PS 4\\Ph 20}

\begin{document}

\section{Part I}
I used git to track the files. \texttt{git init} to create a git
repository, \texttt{git add .} to tell git to track all the files,
\texttt{git commit -a} to commit changes, \texttt{git clone} to make a
copy of the repository somewhere else.

\section{Part II}
The python code for PS 3 was split into 2 files: one for doing all the
calculations and saving the data into text files; the other for reading
the data in the text files and making plots. The data was stored in
\texttt{data/} and the pdf plots were saved in \texttt{latex/}

A makefile was written to run the python codes if needed and to update
the writeup (\texttt{pdflatex}). A separate makefile, \texttt{Makelatex}
which I wrote a few years ago and have been using for all my latex
documents, handled all the latex stuff, and was called using
\texttt{make -f Makelatex} in the main makefile.

\subsection{Makefile}
\lstset{language = [gnu]make}
\lstinputlisting{../../3/Makefile}

\subsection{Makefile for latex things}
(Random story: I wrote this my freshman year and was the first makefile
I ever wrote.  I showed it to a friend, Chi Feng, and he said it looked
ok and that made me happy and hence the first line tag I've kept in it
since then.)
\lstset{language = [gnu]make}
\lstinputlisting{../../3/Makelatex}

\subsection{Python code for generating data}
\lstinputlisting[language = python]{../../3/ps3.py}

\subsection{Python code for making plots}
\lstinputlisting[language = python]{../../3/ps3.py}

\end{document}
