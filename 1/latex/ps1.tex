\documentclass[12pt, oneside, letterpaper, fleqn]{article}

\usepackage{ducky}

\pagestyle{fancy}
\headheight 1.36cm
\fancyhead[L]{Bryance Oyang\\PS 1\\Ph 20}

\newcommand{\w}{\omega}

\renewcommand{\labelenumi}{(\alph{enumi})}

\begin{document}

\paragraph{Problem 2}
\texttt{
fx:\. 1\,
fy:\. 2\,
Ax:\. 1\,
Ay:\. 1\,
phi:\. 1\,
Delta t:\. 0.01\,
N:\. 100\,}
\begin{figure}[htbp]
\includegraphics[width=0.8\textwidth]{plot_2_numpy.pdf}
\end{figure}

\paragraph{Problem 3}
\begin{enumerate}
\item Figure will repeat $\iff$ $f_x t =$ integer and $f_y t =$ integer
for some nonzero $t$ $\iff$ $\frac{f_x}{f_y} =$ rational, obtained by
dividing the two previous equations. 

Finding smallest positive $t$ when it repeats: let $f_y = \frac{a}{b}
f_x$ where $a,b$ are integers, so that the ratio between $f_x$ and $f_y$
is rational. $f_x t =$ integer $\iff t = \frac{n}{f_x}$ where $n$ is an
integer. $f_y \frac{n}{f_x} = \frac{a}{b} n$ is an integer if $an$ is
divisible by $b$. If we want to look for the smallest positive $n$ where
$an$ is divisible by $b$, we must have $an = \lcm(a, b)$ because $an$ is
by default already divisible by $a$. $an = \lcm(a, b) =
\frac{ab}{\gcd(a, b)}$ so $n = \frac{b}{\gcd(a, b)}$. So the earliest
positive time that the figure repeats is $t = \frac{b}{\gcd(a, b) f_x} =
\frac{a}{\gcd(a, b) f_y}$. This simplifies if we assume $\frac{a}{b}$ is
in lowest terms so that $\gcd(a, b) = 1$.

\item Lissajous Figures: Ratio $= f_x/f_y$. Adjusts number of loopies in
horizontal/vertical.
\begin{figure}[htbp]
\includegraphics[width=0.8\textwidth]{plot_3_ratio_1_2_phase_0.pdf}
\end{figure}
\begin{figure}[htbp]
\includegraphics[width=0.8\textwidth]{plot_3_ratio_2_3_phase_0.pdf}
\end{figure}
\begin{figure}[htbp]
\includegraphics[width=0.8\textwidth]{plot_3_ratio_3_4_phase_0.pdf}
\end{figure}

\pagebreak
\item Can tune oscilloscope by making it a straight line with no
deviations. No deviations makes sure the frequency is same for both, and
straight line makes phase the same.

$\phi$ adjusts how circly it is
\begin{figure}[htbp]
\includegraphics[width=0.8\textwidth]{plot_3_ratio_1_1_phase_0.pdf}
\end{figure}
\begin{figure}[htbp]
\includegraphics[width=0.8\textwidth]{plot_3_ratio_1_1_phase_1.pdf}
\end{figure}
\begin{figure}[htbp]
\includegraphics[width=0.8\textwidth]{plot_3_ratio_1_1_phase_2.pdf}
\end{figure}
\begin{figure}[htbp]
\includegraphics[width=0.8\textwidth]{plot_3_ratio_1_1_phase_3.pdf}
\end{figure}
\end{enumerate}

\pagebreak
\paragraph{Problem 4}
Beats. We have something like $\sin((\w_1 - \w_2)t/2)\sin((\w_1 +
\w_2)t/2)$. The amplitude of the $(\w_1 + \w_2)/2$ part is controlled by
$(\w_1 - \w_2)/2$, but the beats show up with frequency $(\w_1 - \w_2)$
because negative amplitude just gives the upside down of the positive,
which looks almost the same.
\begin{figure}[htbp]
\includegraphics[width=0.8\textwidth]{plot_4.pdf}
\end{figure}

\paragraph{Problem 5}
First time officially pythoning. Numpy's $\texttt{savetxt}$ makes life
easier compared to C. Pyplot's plots are very pretty by default.

I'd prefer python over matlab.

\end{document}
